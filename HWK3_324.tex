% Options for packages loaded elsewhere
\PassOptionsToPackage{unicode}{hyperref}
\PassOptionsToPackage{hyphens}{url}
%
\documentclass[
]{article}
\usepackage{amsmath,amssymb}
\usepackage{lmodern}
\usepackage{iftex}
\ifPDFTeX
  \usepackage[T1]{fontenc}
  \usepackage[utf8]{inputenc}
  \usepackage{textcomp} % provide euro and other symbols
\else % if luatex or xetex
  \usepackage{unicode-math}
  \defaultfontfeatures{Scale=MatchLowercase}
  \defaultfontfeatures[\rmfamily]{Ligatures=TeX,Scale=1}
\fi
% Use upquote if available, for straight quotes in verbatim environments
\IfFileExists{upquote.sty}{\usepackage{upquote}}{}
\IfFileExists{microtype.sty}{% use microtype if available
  \usepackage[]{microtype}
  \UseMicrotypeSet[protrusion]{basicmath} % disable protrusion for tt fonts
}{}
\makeatletter
\@ifundefined{KOMAClassName}{% if non-KOMA class
  \IfFileExists{parskip.sty}{%
    \usepackage{parskip}
  }{% else
    \setlength{\parindent}{0pt}
    \setlength{\parskip}{6pt plus 2pt minus 1pt}}
}{% if KOMA class
  \KOMAoptions{parskip=half}}
\makeatother
\usepackage{xcolor}
\usepackage[margin=1in]{geometry}
\usepackage{color}
\usepackage{fancyvrb}
\newcommand{\VerbBar}{|}
\newcommand{\VERB}{\Verb[commandchars=\\\{\}]}
\DefineVerbatimEnvironment{Highlighting}{Verbatim}{commandchars=\\\{\}}
% Add ',fontsize=\small' for more characters per line
\usepackage{framed}
\definecolor{shadecolor}{RGB}{248,248,248}
\newenvironment{Shaded}{\begin{snugshade}}{\end{snugshade}}
\newcommand{\AlertTok}[1]{\textcolor[rgb]{0.94,0.16,0.16}{#1}}
\newcommand{\AnnotationTok}[1]{\textcolor[rgb]{0.56,0.35,0.01}{\textbf{\textit{#1}}}}
\newcommand{\AttributeTok}[1]{\textcolor[rgb]{0.77,0.63,0.00}{#1}}
\newcommand{\BaseNTok}[1]{\textcolor[rgb]{0.00,0.00,0.81}{#1}}
\newcommand{\BuiltInTok}[1]{#1}
\newcommand{\CharTok}[1]{\textcolor[rgb]{0.31,0.60,0.02}{#1}}
\newcommand{\CommentTok}[1]{\textcolor[rgb]{0.56,0.35,0.01}{\textit{#1}}}
\newcommand{\CommentVarTok}[1]{\textcolor[rgb]{0.56,0.35,0.01}{\textbf{\textit{#1}}}}
\newcommand{\ConstantTok}[1]{\textcolor[rgb]{0.00,0.00,0.00}{#1}}
\newcommand{\ControlFlowTok}[1]{\textcolor[rgb]{0.13,0.29,0.53}{\textbf{#1}}}
\newcommand{\DataTypeTok}[1]{\textcolor[rgb]{0.13,0.29,0.53}{#1}}
\newcommand{\DecValTok}[1]{\textcolor[rgb]{0.00,0.00,0.81}{#1}}
\newcommand{\DocumentationTok}[1]{\textcolor[rgb]{0.56,0.35,0.01}{\textbf{\textit{#1}}}}
\newcommand{\ErrorTok}[1]{\textcolor[rgb]{0.64,0.00,0.00}{\textbf{#1}}}
\newcommand{\ExtensionTok}[1]{#1}
\newcommand{\FloatTok}[1]{\textcolor[rgb]{0.00,0.00,0.81}{#1}}
\newcommand{\FunctionTok}[1]{\textcolor[rgb]{0.00,0.00,0.00}{#1}}
\newcommand{\ImportTok}[1]{#1}
\newcommand{\InformationTok}[1]{\textcolor[rgb]{0.56,0.35,0.01}{\textbf{\textit{#1}}}}
\newcommand{\KeywordTok}[1]{\textcolor[rgb]{0.13,0.29,0.53}{\textbf{#1}}}
\newcommand{\NormalTok}[1]{#1}
\newcommand{\OperatorTok}[1]{\textcolor[rgb]{0.81,0.36,0.00}{\textbf{#1}}}
\newcommand{\OtherTok}[1]{\textcolor[rgb]{0.56,0.35,0.01}{#1}}
\newcommand{\PreprocessorTok}[1]{\textcolor[rgb]{0.56,0.35,0.01}{\textit{#1}}}
\newcommand{\RegionMarkerTok}[1]{#1}
\newcommand{\SpecialCharTok}[1]{\textcolor[rgb]{0.00,0.00,0.00}{#1}}
\newcommand{\SpecialStringTok}[1]{\textcolor[rgb]{0.31,0.60,0.02}{#1}}
\newcommand{\StringTok}[1]{\textcolor[rgb]{0.31,0.60,0.02}{#1}}
\newcommand{\VariableTok}[1]{\textcolor[rgb]{0.00,0.00,0.00}{#1}}
\newcommand{\VerbatimStringTok}[1]{\textcolor[rgb]{0.31,0.60,0.02}{#1}}
\newcommand{\WarningTok}[1]{\textcolor[rgb]{0.56,0.35,0.01}{\textbf{\textit{#1}}}}
\usepackage{longtable,booktabs,array}
\usepackage{calc} % for calculating minipage widths
% Correct order of tables after \paragraph or \subparagraph
\usepackage{etoolbox}
\makeatletter
\patchcmd\longtable{\par}{\if@noskipsec\mbox{}\fi\par}{}{}
\makeatother
% Allow footnotes in longtable head/foot
\IfFileExists{footnotehyper.sty}{\usepackage{footnotehyper}}{\usepackage{footnote}}
\makesavenoteenv{longtable}
\usepackage{graphicx}
\makeatletter
\def\maxwidth{\ifdim\Gin@nat@width>\linewidth\linewidth\else\Gin@nat@width\fi}
\def\maxheight{\ifdim\Gin@nat@height>\textheight\textheight\else\Gin@nat@height\fi}
\makeatother
% Scale images if necessary, so that they will not overflow the page
% margins by default, and it is still possible to overwrite the defaults
% using explicit options in \includegraphics[width, height, ...]{}
\setkeys{Gin}{width=\maxwidth,height=\maxheight,keepaspectratio}
% Set default figure placement to htbp
\makeatletter
\def\fps@figure{htbp}
\makeatother
\setlength{\emergencystretch}{3em} % prevent overfull lines
\providecommand{\tightlist}{%
  \setlength{\itemsep}{0pt}\setlength{\parskip}{0pt}}
\setcounter{secnumdepth}{-\maxdimen} % remove section numbering
\ifLuaTeX
  \usepackage{selnolig}  % disable illegal ligatures
\fi
\IfFileExists{bookmark.sty}{\usepackage{bookmark}}{\usepackage{hyperref}}
\IfFileExists{xurl.sty}{\usepackage{xurl}}{} % add URL line breaks if available
\urlstyle{same} % disable monospaced font for URLs
\hypersetup{
  pdftitle={Stat 324 Homework \#3 Due Wednesday February 15th 9am},
  hidelinks,
  pdfcreator={LaTeX via pandoc}}

\title{Stat 324 Homework \#3 Due Wednesday February 15th 9am}
\author{}
\date{\vspace{-2.5em}}

\begin{document}
\maketitle

*Submit your homework to Canvas by the due date and time. Email your
lecturer if you have extenuating circumstances and need to request an
extension.

*If an exercise asks you to use R, include a copy of the code and
output. Please edit your code and output to be only the relevant
portions.

*If a problem does not specify how to compute the answer, you many use
any appropriate method. I may ask you to use R or use manually
calculations on your exams, so practice accordingly.

*You must include an explanation and/or intermediate calculations for an
exercise to be complete.

*Be sure to submit the HWK3 Auto grade Quiz which will give you
\textasciitilde20 of your 40 accuracy points.

*50 points total: 40 points accuracy, and 10 points completion

\vspace{.5cm}

\textbf{Exercise 1:} A chemical supply company ships a certain solvent
in 10-gallon drums. Let X represent the number of drums ordered by a
randomly chosen customer. Assume X has the following probability mass
function (pmf). The mean and variance of X is : \(\mu_X=2.2\) and
\(\sigma^2_X=1.76=1.32665^2\):

\begin{longtable}[]{@{}cc@{}}
\toprule()
X & P(X=x) \\
\midrule()
\endhead
1 & 0.4 \\
2 & 0.3 \\
3 & 0.1 \\
4 & 0.1 \\
5 & 0.1 \\
\bottomrule()
\end{longtable}

\begin{quote}
\begin{enumerate}
\def\labelenumi{\alph{enumi}.}
\tightlist
\item
  Calculate \(P(X \le 2)\) and describe what it means in the context of
  the problem.
\end{enumerate}
\end{quote}

\begin{quote}
\begin{enumerate}
\def\labelenumi{\alph{enumi}.}
\setcounter{enumi}{1}
\tightlist
\item
  Let Y be the number of gallons ordered, so \(Y=10X\). Find the
  probability mass function of Y.
\end{enumerate}
\end{quote}

\begin{longtable}[]{@{}cc@{}}
\toprule()
y & P(Y=y) \\
\midrule()
\endhead
& \\
& \\
\bottomrule()
\end{longtable}

\begin{quote}
\begin{enumerate}
\def\labelenumi{\alph{enumi}.}
\setcounter{enumi}{2}
\tightlist
\item
  Calculate \(\mu_Y\). Interpret this value.
\end{enumerate}
\end{quote}

\begin{quote}
\begin{enumerate}
\def\labelenumi{\alph{enumi}.}
\setcounter{enumi}{3}
\tightlist
\item
  Calculate \(\sigma_Y\). Interpret this value.
\end{enumerate}
\end{quote}

\vspace{.5cm}

\textbf{Exercise 2} A customer receives a very large shipment of items.
The customer assumes 15\% of the items in the shipment are defective.
You can assume that the defectiveness of items is independent within the
shipment and use a 0.15 probability of defectiveness for each item.

Someone on the quality assurance team samples 4 items. Let X be the
random variable for the number of defective items in the sample.

\begin{quote}
\begin{enumerate}
\def\labelenumi{\alph{enumi}.}
\tightlist
\item
  Determine the probability distribution of X (write out the pmf) using
  probability theory.
\end{enumerate}
\end{quote}

\begin{longtable}[]{@{}cc@{}}
\toprule()
x & P(X=x) \\
\midrule()
\endhead
& \\
& \\
\bottomrule()
\end{longtable}

\begin{quote}
\begin{enumerate}
\def\labelenumi{\alph{enumi}.}
\setcounter{enumi}{1}
\tightlist
\item
  Compute P(X\textgreater0). What does this value mean in the context of
  the scenerio?
\end{enumerate}
\end{quote}

\begin{quote}
\begin{enumerate}
\def\labelenumi{\alph{enumi}.}
\setcounter{enumi}{2}
\tightlist
\item
  What is the expected value for X, \(\mu_X\)? What does that value mean
  in the context of the scenerio?
\end{enumerate}
\end{quote}

\begin{quote}
\begin{enumerate}
\def\labelenumi{\alph{enumi}.}
\setcounter{enumi}{3}
\tightlist
\item
  What is the standard deviation for X, \(\sigma_X\)?
\end{enumerate}
\end{quote}

\begin{quote}
\begin{enumerate}
\def\labelenumi{\alph{enumi}.}
\setcounter{enumi}{4}
\tightlist
\item
  Update the following simulation to check your answers for part 2a.
  (You'll need to also set eval=TRUE for the code to run when you knit.)
  Some questions to consider: Why did I define IsDefective as I did?
  What values wouuld be helpful stored into the CountDefective vector?
  What does the histogram show?
\end{enumerate}
\end{quote}

\begin{Shaded}
\begin{Highlighting}[]
\NormalTok{IsDefective}\OtherTok{=}\FunctionTok{c}\NormalTok{(}\FunctionTok{rep}\NormalTok{(}\DecValTok{1}\NormalTok{,}\DecValTok{15}\NormalTok{), }\FunctionTok{rep}\NormalTok{(}\DecValTok{0}\NormalTok{,}\DecValTok{85}\NormalTok{))}
\NormalTok{manytimes}\OtherTok{=}\FloatTok{100000.} \CommentTok{\#may want to start with manytimes being smaller}
\NormalTok{CountDefective}\OtherTok{=}\FunctionTok{rep}\NormalTok{(}\DecValTok{0}\NormalTok{,manytimes)}
\FunctionTok{set.seed}\NormalTok{(}\DecValTok{1}\NormalTok{)}
\ControlFlowTok{for}\NormalTok{ (i }\ControlFlowTok{in} \DecValTok{1}\SpecialCharTok{:}\NormalTok{manytimes)\{}
\NormalTok{  samp}\OtherTok{=}\DocumentationTok{\#\#\#UPDATE THIS\#\#\#\#}
\NormalTok{  CountDefective[i]}\OtherTok{=}\DocumentationTok{\#\#\#UPDATE THIS\#\#\#\#}
\NormalTok{\}}

\FunctionTok{hist}\NormalTok{(CountDefective, }\AttributeTok{labels=}\ConstantTok{TRUE}\NormalTok{,}
     \AttributeTok{ylim=}\FunctionTok{c}\NormalTok{(}\DecValTok{0}\NormalTok{,.}\DecValTok{7}\SpecialCharTok{*}\NormalTok{manytimes), }\AttributeTok{breaks=}\FunctionTok{seq}\NormalTok{(}\SpecialCharTok{{-}}\FloatTok{0.5}\NormalTok{, }\FloatTok{4.5}\NormalTok{, }\DecValTok{1}\NormalTok{))}

\NormalTok{(}\AttributeTok{probX0=}\DocumentationTok{\#\#\#UPDATE THIS\#\#\#\#)}
\NormalTok{(}\AttributeTok{probX1=}\DocumentationTok{\#\#\#UPDATE THIS\#\#\#\#)}
\DocumentationTok{\#\#\#UPDATE THIS\#\#\#\#}
\DocumentationTok{\#\#\#UPDATE THIS\#\#\#\#}
\end{Highlighting}
\end{Shaded}

\begin{quote}
\begin{enumerate}
\def\labelenumi{\alph{enumi}.}
\setcounter{enumi}{5}
\tightlist
\item
  Suppose the quality assurance employee is now going to look at 20
  items from the shipment. They still believe it is reasonable to use a
  Binomial model (n=20, \(\pi=0.15\)) to describe the number of items in
  those 20 that will have a defect.
\end{enumerate}
\end{quote}

\begin{quote}
\begin{quote}
fi. What the the probability that exactly 5 of those 20 items have a
defect?
\end{quote}
\end{quote}

\begin{quote}
\begin{quote}
fii. What the the probability that 5 or more of those 20 items have a
defect?
\end{quote}
\end{quote}

\begin{quote}
\begin{quote}
fiii. Which histogram given below correctly shows the pdf for the
binomial model described in f?
\end{quote}
\end{quote}

\includegraphics{HWK3_324_files/figure-latex/unnamed-chunk-2-1.pdf}

\vspace{.5 cm}

\textbf{Exercise 3} For each of the following questions, say whether the
random variable is reasonably approximated by a binomial random variable
or not, and explain your answer. Comment on the reasonableness of each
of things that must be true for a variable to be a binomial random
variable (ex: identify \(n:\) the number of Bernoulli trials, \(\pi\)
the probability of success, etc).

\begin{quote}
\begin{enumerate}
\def\labelenumi{\alph{enumi}.}
\tightlist
\item
  A fair die is rolled until a 1 appears, and X denotes the number of
  rolls.
\end{enumerate}
\end{quote}

\begin{quote}
\begin{enumerate}
\def\labelenumi{\alph{enumi}.}
\setcounter{enumi}{1}
\tightlist
\item
  Twenty of the different Badger basketball players each attempt 1 free
  throw and X is the total number of successful attempts.
\end{enumerate}
\end{quote}

\begin{quote}
\begin{enumerate}
\def\labelenumi{\alph{enumi}.}
\setcounter{enumi}{2}
\tightlist
\item
  A die is rolled 40 times. Let X be the face that lands up.
\end{enumerate}
\end{quote}

\begin{quote}
\begin{enumerate}
\def\labelenumi{\alph{enumi}.}
\setcounter{enumi}{3}
\tightlist
\item
  In a bag of 10 batteries, I know 2 are old. Let X be the number of old
  batteries I choose when taking a sample of 4 to put into my
  calculator.
\end{enumerate}
\end{quote}

\begin{quote}
\begin{enumerate}
\def\labelenumi{\alph{enumi}.}
\setcounter{enumi}{4}
\tightlist
\item
  It is reported that 20\% of Madison homeowners have installed a home
  security system. Let X be the number of homes without home security
  systems installed in a random sample of 100 houses in the Madison city
  limits.
\end{enumerate}
\end{quote}

\vspace{.5cm}

\textbf{Exercise 4:} The bonding strength \(S\) of a drop of plastic
glue from a particular manufacturer is thought to be well approximated
by a normal distribution with mean 98 lbs and standard deviation 7.5
lbs. \(S~\sim N(98, 7.5^2)\). Compute the following values using a
normal model assumption.

\begin{quote}
\begin{enumerate}
\def\labelenumi{\alph{enumi}.}
\tightlist
\item
  What proportion of drops of plastic glue will have a bonding strength
  between 95 and 104 lbs according to this model?
\end{enumerate}
\end{quote}

\begin{quote}
\begin{enumerate}
\def\labelenumi{\alph{enumi}.}
\setcounter{enumi}{1}
\tightlist
\item
  A single drop of that glue had a bonding strength that is 0.5 standard
  deviations above the mean. What proportion of glue drops have a
  bonding strength that is higher ?
\end{enumerate}
\end{quote}

\begin{quote}
\begin{enumerate}
\def\labelenumi{\alph{enumi}.}
\setcounter{enumi}{2}
\tightlist
\item
  What bonding strength did a drop of glue have that is at the 90th
  percentile?
\end{enumerate}
\end{quote}

\begin{quote}
\begin{enumerate}
\def\labelenumi{\alph{enumi}.}
\setcounter{enumi}{3}
\tightlist
\item
  What is the IQR of bonding strength for drops of glue from this
  manufacturer?
\end{enumerate}
\end{quote}

\begin{quote}
\begin{enumerate}
\def\labelenumi{\alph{enumi}.}
\setcounter{enumi}{4}
\tightlist
\item
  Drops of a similar plastic glue from another manufacturer
  (manufacturer B) is claimed to have bonding strength well approximated
  by a normal distribution with mean 43 kg and standard deviation of 3.5
  kg \(W_{B.kg}~\sim N(43, 3.5^2)\). What is the probability that a drop
  of manufacturer B's glue will have strength above the 90th percentile
  strength of manufacturer A's glue? You can use the conversion: 1 kg
  \(\approx\) 2.20462 lbs.
\end{enumerate}
\end{quote}

\end{document}
